\paragraph{} In this paper we will be discussing the process of using drones to service hospitals of Puerto Rico after Hurricane Maria ravaged its infrastructure. This is a proposal for Help Inc. on how they can analytically best service five given hospitals in the northeast of Puerto Rico.
\paragraph{} Firstly we define the problem as such: We have eight types of drones, five hospitals to service, and three shipping containers. With these givens, we need to determine the most efficient way to get hospitals their care packages for the most days possible. We also have the second metric of collecting road damage surveillance, however jointly maximizing for days and road surveillance requires us to assign a constant ratio between these two necessities, which is difficult as the two are not directly comparable. Therefore we approach the problem by first maximizing the number of days serviceable and then accounting for road surveillance.
\paragraph{} In order to approach this problem statement we would need key general assumptions for the entire process. The key assumptions we make are that we can only charge drones at our home base(s) and that packing ability for shipping containers is decided by volume, not optimizing packing orientations. As there is no good algorithm for 3D-space optimization, this assumption is used to lighten the math.
\paragraph{} When coming up with our solution our approach resembled that of following logical reasoning and then brute forcing a computational solution. For instance in our base model we started by making further assumptions such as we can place home bases anywhere and package load of the drones will not impact their flying ability. These two assumptions are not likely to be true, however we needed a starting block. We then proceeded to logically eliminate drones, base locations, and predict expected results. We then proceeded to brute force a solution through a simulation that computes the optimal locations for our bases. We found that there are many different possibilities of bases that could service all of the hospitals so the next logical step was to grade them based on their road surveillance performance. After computationally comparing potential road reconnaissance of the base locations to each other we came across an optimal location for the three bases. We then needed to divide up our supplies into containers that would take the least amount of time to deliver supplies to the hospitals. Our last step in our base model was to create a flight schedule for the drones each day and specify what modifications will be needed when as container supplies become depleted.
\paragraph{} We conclude without a definite answer as there are a lot of further variables to consider. In our conclusion we acknowledge that some of our assumptions were unrealistic and would need a further report to explore. However, we were able to provide HELP Inc. with a base model of how they could service Puerto Rico under ideal conditions. 